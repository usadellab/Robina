\HeaderA{pseudo.coloring}{Coloring pseudo chip images}{pseudo.coloring}
\aliasA{pseudoColorBar}{pseudo.coloring}{pseudoColorBar}
\aliasA{pseudoPalette}{pseudo.coloring}{pseudoPalette}
\keyword{manip}{pseudo.coloring}
\begin{Description}\relax
These are routines used for coloring pseudo chip images.
\end{Description}
\begin{Usage}
\begin{verbatim}
  pseudoPalette(low = "white", high = c("green", "red"), mid = NULL,k =50) 
  pseudoColorBar(x, horizontal = TRUE, col = heat.colors(50), scale = 1:length(x),k = 11, log.ticks=FALSE,...)
  
\end{verbatim}
\end{Usage}
\begin{Arguments}
\begin{ldescription}
\item[\code{low}] color at low end of scale
\item[\code{high}] color at high end of scale
\item[\code{mid}] color at exact middle of scale
\item[\code{k}] number of colors to have
\item[\code{x}] A data series
\item[\code{horizontal}] If \code{TRUE} then color bar is to be draw
horizontally
\item[\code{col}] colors for color bar
\item[\code{scale}] tickmarks for \code{x} if \code{x} is not numeric
\item[\code{log.ticks}] use a log type transformation to assign the colors
\item[\code{...}] additional parameters to plotting routine
\end{ldescription}
\end{Arguments}
\begin{Details}\relax
Adapted from similar tools in maPlots pacakge.
\end{Details}
\begin{Author}\relax
Ben Bolstad \email{bmb@bmbolstad.com}
\end{Author}
\begin{SeeAlso}\relax
\code{\LinkA{AffyBatch}{AffyBatch}}, \code{\LinkA{read.affybatch}{read.affybatch}}
\end{SeeAlso}

