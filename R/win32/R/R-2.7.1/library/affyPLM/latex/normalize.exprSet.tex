\HeaderA{normalize.ExpressionSet}{Normalization applied to ExpressionSets}{normalize.ExpressionSet}
\methaliasA{normalize.ExpressionSet.contrasts}{normalize.ExpressionSet}{normalize.ExpressionSet.contrasts}
\methaliasA{normalize.ExpressionSet.invariantset}{normalize.ExpressionSet}{normalize.ExpressionSet.invariantset}
\methaliasA{normalize.ExpressionSet.loess}{normalize.ExpressionSet}{normalize.ExpressionSet.loess}
\methaliasA{normalize.ExpressionSet.methods}{normalize.ExpressionSet}{normalize.ExpressionSet.methods}
\methaliasA{normalize.ExpressionSet.qspline}{normalize.ExpressionSet}{normalize.ExpressionSet.qspline}
\methaliasA{normalize.ExpressionSet.quantiles}{normalize.ExpressionSet}{normalize.ExpressionSet.quantiles}
\methaliasA{normalize.ExpressionSet.scaling}{normalize.ExpressionSet}{normalize.ExpressionSet.scaling}
\keyword{manip}{normalize.ExpressionSet}
\begin{Description}\relax
Allows the user to apply normalization routines to ExpressionSets.
\end{Description}
\begin{Usage}
\begin{verbatim}
  normalize.ExpressionSet.quantiles(eset,transfn=c("none","log","antilog"))
  normalize.ExpressionSet.loess(eset,transfn=c("none","log","antilog"),...)
  normalize.ExpressionSet.contrasts(eset, span = 2/3, choose.subset = TRUE, subset.size = 5000, verbose = TRUE, family = "symmetric",transfn=c("none","log","antilog"))
  normalize.ExpressionSet.qspline(eset,transfn=c("none","log","antilog"),...)
  normalize.ExpressionSet.invariantset(eset,prd.td = c(0.003, 0.007),verbose=FALSE,transfn=c("none","log","antilog"),baseline.type=c("mean","median","pseudo-mean","pseudo-median"))
normalize.ExpressionSet.scaling(eset,trim=0.02,baseline=-1,transfn=c("none","log","antilog"))
\end{verbatim}
\end{Usage}
\begin{Arguments}
\begin{ldescription}
\item[\code{eset}] An \code{\LinkA{ExpressionSet}{ExpressionSet}}
\item[\code{span}] parameter to be passed to the function \code{\LinkA{loess}{loess}}.
\item[\code{choose.subset}] 
\item[\code{subset.size}] 
\item[\code{verbose}] verbosity flag
\item[\code{family}] parameter to be passed to the function
\code{\LinkA{loess}{loess}}.
\item[\code{prd.td}] cutoff parameter (details in the bibliographic
reference)
\item[\code{trim}] How much to trim from the top and bottom before computing
the mean when using the scaling normalization
\item[\code{baseline}] Index of array to use as baseline, negative values
(-1,-2,-3,-4) control different baseline selection methods
\item[\code{transfn}] Transform the ExpressionSet before normalizing. Useful when
dealing with expression values that are log-scale
\item[\code{baseline.type}] A method of selecting the baseline array
\item[\code{...}] Additional parameters that may be passed to the
normalization routine
\end{ldescription}
\end{Arguments}
\begin{Details}\relax
This function carries out normalization of expression values. In
general you should either normalize at the probe level or at the
expression value level, not both.

Typing \code{normalize.ExpressionSet.methods} should give you a list of
methods that you may use. note that you can also use the
\code{normalize} function on ExpressionSets. Use \code{method} to select the
normalization method.
\end{Details}
\begin{Value}
A normalized \code{\LinkA{ExpressionSet}{ExpressionSet}}.
\end{Value}
\begin{Author}\relax
Ben Bolstad, \email{bmb@bmbolstad.com}
\end{Author}
\begin{References}\relax
Bolstad, BM (2004) \emph{Low Level Analysis of High-density
Oligonucleotide Array Data: Background, Normalization and
Summarization}. PhD Dissertation. University of California,
Berkeley.
\end{References}
\begin{SeeAlso}\relax
\code{\LinkA{normalize}{normalize}}
\end{SeeAlso}
\begin{Examples}
\begin{ExampleCode}
data(affybatch.example)
eset <- rma(affybatch.example,normalize=FALSE,background=FALSE)
normalize(eset)
\end{ExampleCode}
\end{Examples}

