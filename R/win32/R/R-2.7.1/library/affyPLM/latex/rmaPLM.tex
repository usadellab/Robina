\HeaderA{rmaPLM}{Fit a RMA to Affymetrix Genechip Data as a PLMset}{rmaPLM}
\keyword{manip}{rmaPLM}
\begin{Description}\relax
This function converts an \code{\LinkA{AffyBatch}{AffyBatch}} into an
\code{\LinkA{PLMset}{PLMset}} by fitting a multichip model. In particular we
concentrate on the RMA model.
\end{Description}
\begin{Usage}
\begin{verbatim}
rmaPLM(object,subset=NULL,normalize=TRUE,background=TRUE,background.method="RMA.2",normalize.method="quantile",background.param = list(),normalize.param=list(),output.param=list(),model.param=list(),verbosity.level=0)
\end{verbatim}
\end{Usage}
\begin{Arguments}
\begin{ldescription}
\item[\code{object}] an \code{\LinkA{AffyBatch}{AffyBatch}}
\item[\code{subset}] a vector with the names of probesets to be used. If NULL
then all probesets are used.
\item[\code{normalize}] logical value. If \code{TRUE} normalize data using
quantile normalization
\item[\code{background}] logical value. If \code{TRUE} background correct
using RMA background correction
\item[\code{background.method}] name of background method to use.
\item[\code{normalize.method}] name of normalization method to use.
\item[\code{background.param}] A list of parameters for background routines
\item[\code{normalize.param}] A list of parameters for normalization
routines
\item[\code{output.param}] A list of parameters controlling optional output
from the routine.
\item[\code{model.param}] A list of parameters controlling model procedure
\item[\code{verbosity.level}] An integer specifying how much to print
out. Higher values indicate more verbose. A value of 0 will print nothing
\end{ldescription}
\end{Arguments}
\begin{Details}\relax
This function fits the RMA as a Probe Level Linear models to all the probesets in
an \code{\LinkA{AffyBatch}{AffyBatch}}.
\end{Details}
\begin{Value}
An \code{\LinkA{PLMset}{PLMset}}
\end{Value}
\begin{Author}\relax
Ben Bolstad \email{bmb@bmbolstad.com}
\end{Author}
\begin{References}\relax
Bolstad, BM (2004) \emph{Low Level Analysis of High-density
Oligonucleotide Array Data: Background, Normalization and
Summarization}. PhD Dissertation. University of California,
\\ \\ Irizarry RA, Bolstad BM, Collin F, Cope LM, Hobbs B and Speed
TP (2003) \emph{Summaries of Affymetrix GeneChip probe level data}
Nucleic Acids Research 31(4):e15
\\ \\ Bolstad, BM, Irizarry RA, Astrand, M, and Speed, TP (2003)
\emph{A Comparison of Normalization Methods for High Density
Oligonucleotide Array Data Based on Bias and Variance.}
Bioinformatics 19(2):185-193
\end{References}
\begin{SeeAlso}\relax
\code{\LinkA{expresso}{expresso}},
\code{\LinkA{rma}{rma}}, \code{\LinkA{threestep}{threestep}},\code{\LinkA{fitPLM}{fitPLM}}, \code{\LinkA{threestepPLM}{threestepPLM}}
\end{SeeAlso}
\begin{Examples}
\begin{ExampleCode}
# A larger example testing weight image function
data(Dilution)
## Not run: Pset <- rmaPLM(Dilution,output.param=list(weights=TRUE))
## Not run: image(Pset)
\end{ExampleCode}
\end{Examples}

