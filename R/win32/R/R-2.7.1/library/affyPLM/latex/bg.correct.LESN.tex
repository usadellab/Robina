\HeaderA{bg.correct.LESN}{LESN - Low End Signal is Noise Background corrections}{bg.correct.LESN}
\keyword{manip}{bg.correct.LESN}
\begin{Description}\relax
This function background corrects PM probe data using LESN - Low End
Signal is Noise concepts.
\end{Description}
\begin{Usage}
\begin{verbatim}
bg.correct.LESN(object,method = 2,baseline = 0.25, theta=4)
\end{verbatim}
\end{Usage}
\begin{Arguments}
\begin{ldescription}
\item[\code{object}] an \code{\LinkA{AffyBatch}{AffyBatch}}
\item[\code{method}] an integer code specifying which method to use
\item[\code{baseline}] A baseline value to use
\item[\code{theta}] A parameter used in the background correction process
\end{ldescription}
\end{Arguments}
\begin{Details}\relax
This method will be more formally documented at a later date.

The basic concept is to consider that the lowest end of intensites is
most likely just noise (and should be heavily corrected) and the
highest end signals are most likely signal and should have little
adjustment. Low end signals are made much smaller while high end
signals get less adjustment relative adjustment.
\end{Details}
\begin{Value}
An \code{\LinkA{AffyBatch}{AffyBatch}}
\end{Value}
\begin{Author}\relax
Ben Bolstad \email{bmb@bmbolstad.com}
\end{Author}
\begin{References}\relax
Bolstad, BM (2004) \emph{Low Level Analysis of High-density
Oligonucleotide Array Data: Background, Normalization and
Summarization}. PhD Dissertation. University of California, Berkeley.
\end{References}
\begin{Examples}
\begin{ExampleCode}
data(affybatch.example)
affybatch.example.bgcorrect <- bg.correct.LESN(affybatch.example)
\end{ExampleCode}
\end{Examples}

