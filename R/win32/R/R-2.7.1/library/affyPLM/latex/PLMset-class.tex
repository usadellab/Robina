\HeaderA{PLMset-class}{Class PLMset}{PLMset.Rdash.class}
\aliasA{annotation,PLMset-method}{PLMset-class}{annotation,PLMset.Rdash.method}
\aliasA{boxplot,PLMset-method}{PLMset-class}{boxplot,PLMset.Rdash.method}
\aliasA{cdfName,PLMset-method}{PLMset-class}{cdfName,PLMset.Rdash.method}
\aliasA{coefs}{PLMset-class}{coefs}
\aliasA{coefs,PLMset-method}{PLMset-class}{coefs,PLMset.Rdash.method}
\aliasA{coefs.const}{PLMset-class}{coefs.const}
\aliasA{coefs.const,PLMset-method}{PLMset-class}{coefs.const,PLMset.Rdash.method}
\aliasA{coefs.probe}{PLMset-class}{coefs.probe}
\aliasA{coefs.probe,PLMset-method}{PLMset-class}{coefs.probe,PLMset.Rdash.method}
\aliasA{coefs<\Rdash}{PLMset-class}{coefs<.Rdash.}
\aliasA{coefs<-,PLMset-method}{PLMset-class}{coefs<.Rdash.,PLMset.Rdash.method}
\aliasA{description,PLMset-method}{PLMset-class}{description,PLMset.Rdash.method}
\aliasA{getCdfInfo,PLMset-method}{PLMset-class}{getCdfInfo,PLMset.Rdash.method}
\aliasA{image,PLMset-method}{PLMset-class}{image,PLMset.Rdash.method}
\aliasA{indexProbes,PLMset,character-method}{PLMset-class}{indexProbes,PLMset,character.Rdash.method}
\aliasA{indexProbesProcessed}{PLMset-class}{indexProbesProcessed}
\aliasA{indexProbesProcessed,PLMset-method}{PLMset-class}{indexProbesProcessed,PLMset.Rdash.method}
\aliasA{Mbox}{PLMset-class}{Mbox}
\aliasA{Mbox,PLMset-method}{PLMset-class}{Mbox,PLMset.Rdash.method}
\aliasA{model.description}{PLMset-class}{model.description}
\aliasA{model.description,PLMset-method}{PLMset-class}{model.description,PLMset.Rdash.method}
\aliasA{normvec}{PLMset-class}{normvec}
\aliasA{normvec,PLMset-method}{PLMset-class}{normvec,PLMset.Rdash.method}
\aliasA{NUSE}{PLMset-class}{NUSE}
\aliasA{nuse}{PLMset-class}{nuse}
\aliasA{NUSE,PLMset-method}{PLMset-class}{NUSE,PLMset.Rdash.method}
\aliasA{nuse,PLMset-method}{PLMset-class}{nuse,PLMset.Rdash.method}
\aliasA{pData,PLMset-method}{PLMset-class}{pData,PLMset.Rdash.method}
\aliasA{pData<-,PLMset,data.frame-method}{PLMset-class}{pData<.Rdash.,PLMset,data.frame.Rdash.method}
\aliasA{phenoData,PLMset-method}{PLMset-class}{phenoData,PLMset.Rdash.method}
\aliasA{phenoData<-,PLMset,AnnotatedDataFrame-method}{PLMset-class}{phenoData<.Rdash.,PLMset,AnnotatedDataFrame.Rdash.method}
\aliasA{PLMset}{PLMset-class}{PLMset}
\aliasA{resid,PLMset-method}{PLMset-class}{resid,PLMset.Rdash.method}
\aliasA{resid<\Rdash}{PLMset-class}{resid<.Rdash.}
\aliasA{resid<-,PLMset-method}{PLMset-class}{resid<.Rdash.,PLMset.Rdash.method}
\aliasA{residSE}{PLMset-class}{residSE}
\aliasA{residSE,PLMset-method}{PLMset-class}{residSE,PLMset.Rdash.method}
\aliasA{residuals,PLMset-method}{PLMset-class}{residuals,PLMset.Rdash.method}
\aliasA{residuals<\Rdash}{PLMset-class}{residuals<.Rdash.}
\aliasA{residuals<-,PLMset-method}{PLMset-class}{residuals<.Rdash.,PLMset.Rdash.method}
\aliasA{RLE}{PLMset-class}{RLE}
\aliasA{RLE,PLMset-method}{PLMset-class}{RLE,PLMset.Rdash.method}
\aliasA{sampleNames,PLMset-method}{PLMset-class}{sampleNames,PLMset.Rdash.method}
\aliasA{sampleNames<\Rdash}{PLMset-class}{sampleNames<.Rdash.}
\aliasA{sampleNames<-,PLMset,character-method}{PLMset-class}{sampleNames<.Rdash.,PLMset,character.Rdash.method}
\aliasA{se}{PLMset-class}{se}
\aliasA{se,PLMset-method}{PLMset-class}{se,PLMset.Rdash.method}
\aliasA{se.const}{PLMset-class}{se.const}
\aliasA{se.const,PLMset-method}{PLMset-class}{se.const,PLMset.Rdash.method}
\aliasA{se.probe}{PLMset-class}{se.probe}
\aliasA{se.probe,PLMset-method}{PLMset-class}{se.probe,PLMset.Rdash.method}
\aliasA{se<\Rdash}{PLMset-class}{se<.Rdash.}
\aliasA{se<-,PLMset-method}{PLMset-class}{se<.Rdash.,PLMset.Rdash.method}
\aliasA{show,PLMset-method}{PLMset-class}{show,PLMset.Rdash.method}
\aliasA{summary,PLMset-method}{PLMset-class}{summary,PLMset.Rdash.method}
\aliasA{varcov}{PLMset-class}{varcov}
\aliasA{varcov,PLMset-method}{PLMset-class}{varcov,PLMset.Rdash.method}
\aliasA{weights,PLMset-method}{PLMset-class}{weights,PLMset.Rdash.method}
\aliasA{weights<\Rdash}{PLMset-class}{weights<.Rdash.}
\aliasA{weights<-,PLMset-method}{PLMset-class}{weights<.Rdash.,PLMset.Rdash.method}
\keyword{classes}{PLMset-class}
\begin{Description}\relax
This is a class representation for Probe level Linear
Models fitted to Affymetrix GeneChip probe level data.
\end{Description}
\begin{Section}{Objects from the Class}
Objects can be created using the function \code{\LinkA{fitPLM}{fitPLM}}
\end{Section}
\begin{Section}{Slots}
\describe{
\item[\code{probe.coefs}:] Object of class "matrix". Contains model
coefficients related to probe effects.
\item[\code{se.probe.coefs}:] Object of class "matrix". Contains
standard error estimates for the probe coefficients.  
\item[\code{chip.coefs}:] Object of class "matrix". Contains model
coefficients related to chip (or chip level) effects for each fit.
\item[\code{se.chip.coefs}:] Object of class "matrix". Contains
standard error estimates for the chip coefficients.
\item[\code{model.description}:] Object of class "character". This
string describes the probe level model fitted.
\item[\code{weights}:] List of objects of class "matrix". Contains probe
weights for each fit. The matrix has columns for chips and rows
are probes.
\item[\code{phenoData}:] Object of class "phenoData" This is an
instance of class \code{phenoData} containing the patient
(or case) level data. The columns of the pData slot of this
entity represent variables and the rows represent patients or cases.
\item[\code{annotation}] A character string identifying the
annotation that may be used for the \code{ExpressionSet} instance.
\item[\code{description}:] Object of class "MIAME". For
compatibility with previous version of this class description can
also be a "character". The class \code{characterOrMIAME} has been
defined just for this.
\item[\code{cdfName}:] A character string giving the name of the
cdfFile.
\item[\code{nrow}:] Object of class "numeric". Number of rows in chip.
\item[\code{ncol}:] Object of class "numeric". Number of cols in chip.
\item[\code{notes}:] Object of class "character" Vector of
explanatory text.
\item[\code{varcov}:] Object of class "list". A list of
variance/covariance matrices.
\item[\code{residualSE}:] Object of class "matrix". Contains residual
standard error and df.
\item[\code{residuals}:] List of objects of class "matrix". Contains
residuals from model fit (if stored).
}
\end{Section}
\begin{Section}{Methods}
\describe{
\item[weights<-] \code{signature(object = "PLMset")}: replaces the weights.
\item[weights] \code{signature(object = "PLMset")}: extracts the
model fit weights.
\item[coefs<-] \code{signature(object = "PLMset")}: replaces the
chip coefs.
\item[coefs] \code{signature(object = "PLMset")}: extracts the
chip coefs.
\item[se] \code{signature(object = "PLMset")}: extracts the
standard error estimates of the chip coefs.
\item[se<-] \code{signature(object = "PLMset")}: replaces the
standard error estimates of the chip coefs.
\item[coefs.probe] \code{signature(object = "PLMset")}: extracts the
probe coefs.
\item[se.probe] \code{signature(object = "PLMset")}: extracts the
standard error estimates of the probe coefs.    
\item[coefs.const] \code{signature(object = "PLMset")}: extracts the
intercept coefs.
\item[se.const] \code{signature(object = "PLMset")}: extracts the
standard error estimates of the intercept coefs.
\item[getCdfInfo] \code{signature(object = "PLMset")}: retrieve
the environment that defines the location of probes by probe set.
\item[image] \code{signature(x = "PLMset")}: creates an image
of the robust linear model fit weights for each sample.
\item[indexProbes] \code{signature(object = "PLMset", which =
        "character")}: returns a list with locations of the probes in
each probe  set. The list names defines the probe set
names. \code{which} can be "pm", "mm", or "both". If "both" then
perfect match locations are given followed by mismatch locations.
\item[Mbox] \code{signature(object = "PLMset")}: gives a boxplot of
M's for each chip. The M's are computed relative to a "median"
chip.
\item[normvec] \code{signature(x = "PLMset")}: will return the normalization vector
(if it has been stored).
\item[residSE] \code{signature(x = "PLMset")}: will return the residual SE
(if it has been stored).
\item[boxplot] \code{signature(x = "PLMset")}: Boxplot of Normalized
Unscaled Standard Errors (NUSE).
\item[NUSE] \code{signature(x = "PLMset")} : Boxplot of Normalized
Unscaled Standard Errors (NUSE) or NUSE values.
\item[RLE|] \code{signature(x = "PLMset")} : Relative Log Expression
boxplot or values.
}
\end{Section}
\begin{Note}\relax
This class is better described in the vignette.
\end{Note}
\begin{Author}\relax
B. M. Bolstad \email{bmb@bmbolstad.com}
\end{Author}
\begin{References}\relax
Bolstad, BM (2004) \emph{Low Level Analysis of High-density
Oligonucleotide Array Data: Background, Normalization and
Summarization}. PhD Dissertation. University of California,
Berkeley.
\end{References}

