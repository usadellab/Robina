\HeaderA{normalize.scaling}{Scaling normalization}{normalize.scaling}
\aliasA{normalize.AffyBatch.scaling}{normalize.scaling}{normalize.AffyBatch.scaling}
\keyword{manip}{normalize.scaling}
\begin{Description}\relax
Allows the user to apply scaling normalization.
\end{Description}
\begin{Usage}
\begin{verbatim}
normalize.scaling(X,trim=0.02,baseline=-1,log.scalefactors=FALSE)
normalize.AffyBatch.scaling(abatch,type=c("together","pmonly","mmonly","separate"),trim=0.02,baseline=-1,log.scalefactors=FALSE)
\end{verbatim}
\end{Usage}
\begin{Arguments}
\begin{ldescription}
\item[\code{X}] A matrix. The columns of which are to be normalized.
\item[\code{abatch}] An \code{\LinkA{AffyBatch}{AffyBatch}}
\item[\code{type}] A parameter controlling how normalization is applied to
the Affybatch.
\item[\code{trim}] How much to trim from the top and bottom before computing
the mean when using the scaling normalization.
\item[\code{baseline}] Index of array to use as baseline, negative values
(-1,-2,-3,-4) control different baseline selection methods.
\item[\code{log.scalefactors}] Compute the scale factors based on log2
transformed data.
\end{ldescription}
\end{Arguments}
\begin{Details}\relax
These function carries out scaling normalization of expression values.
\end{Details}
\begin{Value}
A normalized \code{\LinkA{ExpressionSet}{ExpressionSet}}.
\end{Value}
\begin{Author}\relax
Ben Bolstad, \email{bmb@bmbolstad.com}
\end{Author}
\begin{SeeAlso}\relax
\code{\LinkA{normalize}{normalize}}
\end{SeeAlso}
\begin{Examples}
\begin{ExampleCode}
data(affybatch.example)
normalize.AffyBatch.scaling(affybatch.example)
\end{ExampleCode}
\end{Examples}

