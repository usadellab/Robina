\HeaderA{lymphoma}{Subset of the Intensity data for 8 cDNA slides with CLL and DLBL samples
from the Alizadeh et al. paper in Nature 2000}{lymphoma}
\aliasA{lym.exp}{lymphoma}{lym.exp}
\keyword{datasets}{lymphoma}
\begin{Description}\relax
8 cDNA chips from Alizadeh lymphoma paper
\end{Description}
\begin{Usage}
\begin{verbatim}data(lymphoma)\end{verbatim}
\end{Usage}
\begin{Format}\relax
\code{lymphoma} is an \code{\LinkA{exprSet}{exprSet}}
containing the data from 8 chips
from the lymphoma data set by Alizadeh et al. (see references). Each
chip represents two samples: on color channel 1 (CH1, Cy3, green) the
common reference sample, and on color channel 2 (CH2, Cy5, red) the
various disease samples. See \code{pData(lymphoma)}. The 9216x16
matrix \code{exprs(lymphoma)} contains the background-subtracted spot
intensities (CH1I-CH1B and CH2I-CH2B, respectively).
\end{Format}
\begin{Details}\relax
The chip intensity files were downloaded from the Stanford
microarray database. Starting from the link below, this was done by
following the links \emph{Published Data} -> 
\emph{Alizadeh AA, et al. (2000) Nature 403(6769):503-11} -> 
\emph{Data in SMD} -> \emph{Display Data}, and selecting the following 
8 slides:
\Tabular{l}{
lc7b019\\
lc7b047\\
lc7b048\\
lc7b056\\
lc7b057\\
lc7b058\\
lc7b069\\
lc7b070
}
Then, the script \code{makedata.R} from the \code{scripts} subdirectory
of this package was run to generate the \R{} data object.
\end{Details}
\begin{Source}\relax
http://genome-www5.stanford.edu/MicroArray/SMD
\end{Source}
\begin{References}\relax
A. Alizadeh et al., Distinct types of diffuse large B-cell 
lymphoma identified by gene expression profiling. Nature 403(6769):503-11, 
Feb 3, 2000.
\end{References}

