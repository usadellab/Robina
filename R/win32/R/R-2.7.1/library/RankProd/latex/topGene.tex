\HeaderA{topGene}{Output Significant Genes}{topGene}
\keyword{htest}{topGene}
\begin{Description}\relax
Identify differentially expressed genes using 
rank product method
\end{Description}
\begin{Usage}
\begin{verbatim}

     topGene(x,cutoff=NULL,method="pfp",num.gene=NULL,logged=TRUE,logbase=2,gene.names=NULL)\end{verbatim}
\end{Usage}
\begin{Arguments}
\begin{ldescription}
\item[\code{x}] the value returned by the 
function RP, RP.advance or
Rsum.advance 
\item[\code{cutoff}] threshold in pfp used to select genes
\item[\code{method}] 
\item[\code{num.gene}] number of candidate genes of interests, 
if cutoff is provided, this will be ignored
\item[\code{logged}] if "TRUE", data has bee logged, otherwise set it 
to "FALSE"
\item[\code{logbase}] base used when taking log, used to restore the 
fold change.The default value is 2, this will be 
ignored if logged=FALSE
\item[\code{gene.names}] if "NULL", no gene name will be 
attached to the output table
\end{ldescription}
\end{Arguments}
\begin{Value}
Two tables of identified genes with 
gene.index: index of gene in the original data set 
RP/Rsum: Computed rank product/sum for each gene
FC:(class1/class2): Expression Fold change of class 1/ class 2.                   
pfp: estimated pfp for each gene if the gene is used as cutoff point
P.value: estimated p-value for each gene 

Table 1 list genes that are up-regulated under class 2, Table 1 ist 
genes that are down-regulated under class 2,
\end{Value}
\begin{Author}\relax
Fangxin Hong \email{fhong@salk.edu}
\end{Author}
\begin{References}\relax
Breitling, R., Armengaud, P., Amtmann, A., and Herzyk, 
P.(2004) Rank Products: A simple, yet powerful, new method 
to detect differentially regulated genes in
replicated microarray experiments, \emph{FEBS Letter}, 57383-92
\end{References}
\begin{SeeAlso}\relax
\code{\LinkA{plotRP}{plotRP}} \code{\LinkA{RP}{RP}}  
\code{\LinkA{RPadvance}{RPadvance}} \code{\LinkA{RSadvance}{RSadvance}}
\end{SeeAlso}
\begin{Examples}
\begin{ExampleCode}

      # Load the data of Golub et al. (1999). data(golub) 
      # contains a 3051x38 gene expression
      # matrix called golub, a vector of length called golub.cl 
      # that consists of the 38 class labels,
      # and a matrix called golub.gnames whose third column 
      # contains the gene names.
      data(golub)

      #use a subset of data as example, apply the rank 
      #product method
      subset <- c(1:4,28:30)
      #Setting rand=123, to make the results reproducible,

      #identify genes 
      RP.out <- RP(golub[,subset],golub.cl[subset],rand=123)  

      #get two lists of differentially expressed genes 
      #by setting FDR (false discivery rate) =0.05

      table=topGene(RP.out,cutoff=0.05,method="pfp",logged=TRUE,logbase=2,
                   gene.names=golub.gnames[,3])
      table$Table1
      table$Table2

      #using pvalue<0.05
      topGene(RP.out,cutoff=0.05,method="pval",logged=TRUE,logbase=2,
                   gene.names=golub.gnames[,3])

      #by selecting top 10 genes

      topGene(RP.out,num.gene=10,gene.names=golub.gnames[,3])

\end{ExampleCode}
\end{Examples}

