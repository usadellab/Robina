\HeaderA{RSadvance}{Advanced Rank Sum Analysis of Microarray}{RSadvance}
\keyword{htest}{RSadvance}
\begin{Description}\relax
Advance rank sum method to identify differentially 
expressed genes. It is possible to combine data from 
different studies, e.g. data sets generated at different 
laboratories.
\end{Description}
\begin{Usage}
\begin{verbatim}
    RSadvance(data,cl,origin,num.perm=100,logged=TRUE,
                 na.rm=FALSE,gene.names=NULL,plot=FALSE, rand=NULL)
\end{verbatim}
\end{Usage}
\begin{Arguments}
\begin{ldescription}
\item[\code{data}] the data set that should be analyzed. Every 
row of this data set must correspond to a gene.
\item[\code{cl}] a vector containing the class labels of the 
samples. In the two class unpaired case, the label 
of a sample is either 0 (e.g., control group) or 1 
(e.g., case group). For one group data, the label for 
each sample should be 1.
\item[\code{origin}] a vector containing the origin labels of the 
sample. e.g. for 
the data sets generated at multiple laboratories, the label
is the same for samples within one lab and different for samples 
from different labs. 
\item[\code{num.perm}] number of permutations used in the calculation 
of the null density. Default is 'B=100'.
\item[\code{logged}] if "TRUE", data has bee logged, otherwise set 
it to "FALSE"
\item[\code{na.rm}] if 'FALSE' (default), the NA value will not
be used in computing rank. If 'TRUE', the missing 
values will be replaced by the genewise mean of
the non-missing values. Gene will all value missing 
will be assigned "NA"
\item[\code{gene.names}] if "NULL", no gene name will be attached 
to the estimated percentage of false prediction (pfp). 
\item[\code{plot}] If "TRUE", plot the estimated pfp verse the rank 
of each gene
\item[\code{rand}] if specified, the random number generator 
will be put in a  reproducible state.
\end{ldescription}
\end{Arguments}
\begin{Value}
A result of identifying differentially expressed 
genes between two classes. The identification consists of two parts,
the identification of  up-regulated  and down-regulated genes in class 2
compared to class 1, respectively. 

\begin{ldescription}
\item[\code{pfp}] estimated percentage of false positive predictions
(pfp) up to  the position of each gene under two 
identificaiton each
\item[\code{pval}] estimated pvalue for each gene being up- and down-regulated
\item[\code{RSs}] Origina rank-sum (average rank) of each genes
\item[\code{RSrank}] rank of the rank sum of each gene in ascending order
\item[\code{Orirank}] original ranks in each comparison, which is 
used to compute rank sum
\item[\code{AveFC}] fold change of average expression under class 1 over 
that under class 2, if multiple origin, than avraged 
across all origin. log-fold change if data is in log scaled, 
original fold change if data is unlogged. 
\item[\code{all.FC}] fold change of class 1/class 2 under each origin.
log-fold change if data is in log scaled
\end{ldescription}
\end{Value}
\begin{Note}\relax
Percentage of false prediction (pfp), in theory, is 
equivalent of false  discovery rate (FDR), and it is 
possible to be large than 1.

The function looks for up- and down- regulated genes in two
seperate steps, thus two pfps are computed and used to identify 
gene that belong to each group. 

The function is able to deal with single or multiple-orgin 
studies. It is similar to  funcion RP.advance expect a rank
sum is computed instead of rank product. This method 
is more sensitive to individual rank values, while rank 
product is more robust to 
outliers (refer RankProd vignette for details)
\end{Note}
\begin{Author}\relax
Fangxin Hong \email{fhong@salk.edu}
\end{Author}
\begin{SeeAlso}\relax
\code{\LinkA{topGene}{topGene}}   \code{\LinkA{RP}{RP}}  
\code{\LinkA{plotRP}{plotRP}}  \code{\LinkA{RPadvance}{RPadvance}}
\end{SeeAlso}
\begin{Examples}
\begin{ExampleCode}
      
      #Suppose we want to check the consistence of the data 
      #sets generated in two different 
      #labs. For example, we would look for genes that were \
      # measured to be up-regulated in 
      #class 2 at lab 1, but down-regulated in class 2 at lab 2.\
       data(arab)
      arab.cl2 <- arab.cl

      arab.cl2[arab.cl==0 &arab.origin==2] <- 1

      arab.cl2[arab.cl==1 &arab.origin==2] <- 0

      arab.cl2
  ##[1] 0 0 0 1 1 1 1 1 0 0

      #look for genes differentially expressed
      #between hypothetical class 1 and 2
      arab.sub=arab[1:500,] ##using subset for fast computation
      arab.gnames.sub=arab.gnames[1:500]
      Rsum.adv.out <- RSadvance(arab.sub,arab.cl2,arab.origin,
                          num.perm=100,
logged=TRUE,
                          gene.names=arab.gnames.sub,rand=123)

      attributes(Rsum.adv.out)
      
\end{ExampleCode}
\end{Examples}

