\HeaderA{bg.adjust.gcrma}{GCRMA background adjust (internal function)}{bg.adjust.gcrma}
\keyword{manip}{bg.adjust.gcrma}
\begin{Description}\relax
This function performs background adjustment (optical noise and
non-specific binding on an \code{AffyBatch}
project and returns an \code{AffyBatch} object in which the PM
intensities are adjusted.
\end{Description}
\begin{Usage}
\begin{verbatim}
bg.adjust.gcrma(object,affinity.info=NULL,
      affinity.source=c("reference","local"),
      NCprobe=NULL,
      type=c("fullmodel","affinities","mm","constant"),
      k=6*fast+0.5*(1-fast),stretch=1.15*fast+1*(1-fast),correction=1,
      GSB.adjust=TRUE,
      rho=.7,optical.correct=TRUE,verbose=TRUE,fast=TRUE)
\end{verbatim}
\end{Usage}
\begin{Arguments}
\begin{ldescription}
\item[\code{object}] an \code{\LinkA{AffyBatch}{AffyBatch}}
\item[\code{affinity.info}] \code{NULL} or an \code{AffyBatch} containing the
affinities in the \code{exprs} slot. This object can be created
using the function \code{\LinkA{compute.affinities}{compute.affinities}}.
\item[\code{affinity.source}] \code{reference}: use the package internal
Non-specific binding data or \code{local}: use the experimental
data in \code{object}. If \code{local} is chosen, either MM probes or a user-defined
list of probes (see \code{NCprobes}) are used to estimate affinities.
\item[\code{NCprobe}] 
\item[\code{type}] "fullmodel" for sequence and MM model. "affinities" for
sequence information only. "mm" for using MM without sequence
information.
\item[\code{k}] A tuning factor.
\item[\code{stretch}] 
\item[\code{correction}] .
\item[\code{GSB.adjust}] Logical value. If \code{TRUE}, probe effects in specific binding will
be adjusted.
\item[\code{rho}] correlation coefficient of log background intensity in a pair of pm/mm probes. Default=.7
\item[\code{optical.correct}] Logical value. If \code{TRUE}, optical
background correction is performed.
\item[\code{verbose}] Logical value. If \code{TRUE} messages about the progress of
the function is printed.
\item[\code{fast}] Logical value. If \code{TRUE} a faster ad hoc algorithm is
used.
\end{ldescription}
\end{Arguments}
\begin{Details}\relax
The returned value is an \code{AffyBatch} object, in which the PM probe intensities
have been background adjusted. The rest is left the same as the
starting \code{AffyBatch} object.

The tunning factor \code{k} will have different meainngs if one uses
the fast (ad hoc) algorithm or the empirical bayes approach. See Wu
et al. (2003)
\end{Details}
\begin{Value}
An \code{AffyBatch}.
\end{Value}
\begin{Author}\relax
Rafeal Irizarry
\end{Author}
\begin{Examples}
\begin{ExampleCode}
 if(require(affydata) & require(hgu95av2probe) & require(hgu95av2cdf)){
          data(Dilution)
          ai <- compute.affinities(cdfName(Dilution))
          Dil.adj<-bg.adjust.gcrma(Dilution,affinity.info=ai,type="affinities")
     }
\end{ExampleCode}
\end{Examples}

