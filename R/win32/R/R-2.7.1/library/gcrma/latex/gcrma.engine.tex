\HeaderA{gcrma.engine}{GCRMA background adjust engine(internal function)}{gcrma.engine}
\keyword{manip}{gcrma.engine}
\begin{Description}\relax
This function adjust for non-specific binding when all arrays in the
dataset share the same probe affinity information. It takes matrices of PM
probe intensities, MM probe intensities, other negative control probe
intensities(optional) and the associated probe affinities, and return
one matrix of non-specific binding corrected PM probe intensities.
\end{Description}
\begin{Usage}
\begin{verbatim}
gcrma.engine(pms,mms,ncs=NULL,
                         pm.affinities=NULL,mm.affinities=NULL,anc=NULL,
                         type=c("fullmodel","affinities","mm","constant"),
                         k=6*fast+0.5*(1-fast),
                         stretch=1.15*fast+1*(1-fast),correction=1,GSB.adjust=TRUE,rho=0.7,
                         verbose=TRUE,fast=FALSE)
\end{verbatim}
\end{Usage}
\begin{Arguments}
\begin{ldescription}
\item[\code{pms}] The matrix of PM intensities
\item[\code{mms}] The matrix of MM intensities
\item[\code{ncs}] The matrix of negative control probe intensities. When left
as\code{NULL}, the MMs are considered the negative control probes.
\item[\code{pm.affinities}] The vector of PM probe affinities. Note: This can be
shorter than the number of rows in \code{pms} when some probes do not
have sequence information provided.
\item[\code{mm.affinities}] The vector of MM probe affinities.
\item[\code{anc}] The vector of Negative Control probe affinities. This is
ignored if MMs are used as negative controls (\code{ncs=NULL})
\item[\code{type}] "fullmodel" for sequence and MM model. "affinities" for
sequence information only. "mm" for using MM without sequence
information.
\item[\code{k}] A tuning factor.
\item[\code{stretch}] 
\item[\code{correction}] .
\item[\code{GSB.adjust}] Logical value. If \code{TRUE}, probe effects in specific binding will
be adjusted.
\item[\code{rho}] correlation coefficient of log background intensity in a pair of pm/mm probes. Default=.7
\item[\code{verbose}] Logical value. If \code{TRUE} messages about the progress of
the function is printed.
\item[\code{fast}] Logicalvalue. If \code{TRUE} a faster add-hoc algorithm is
used.
\end{ldescription}
\end{Arguments}
\begin{Details}\relax
Note that this expression measure is given to you in log base 2
scale. This differs from most of the other expression measure
methods.

The tunning factor \code{k} will have different meainngs if one uses
the fast (add-hoc) algorithm or the empirical bayes approach. See Wu
et al. (2003)
\end{Details}
\begin{Value}
A matrix of PM intensties.
\end{Value}
\begin{Author}\relax
Rafeal Irizarry \& Zhijin Wu
\end{Author}
\begin{SeeAlso}\relax
gcrma.engine2
\end{SeeAlso}

